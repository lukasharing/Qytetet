\documentclass[8pt, a4paper, titlepage]{article}

%page styles
\usepackage[utf8]{inputenc}
\usepackage{amsmath}
\usepackage[letterpaper, margin=1.5in]{geometry}
\usepackage{amsfonts}
% Code %
\usepackage{listings}
\usepackage{color}
 
\definecolor{codegreen}{rgb}{0,0.6,0}
\definecolor{codegray}{rgb}{0.5,0.5,0.5}
\definecolor{codepurple}{rgb}{0.58,0,0.82}
\definecolor{backcolour}{rgb}{0.95,0.95,0.92}

%gnu%
\usepackage[miktex]{gnuplottex}

\lstdefinestyle{mystyle}{
    backgroundcolor=\color{backcolour},   
    commentstyle=\color{codegreen},
    keywordstyle=\color{magenta},
    numberstyle=\tiny\color{codegray},
    stringstyle=\color{codepurple},
    basicstyle=\footnotesize,
    breakatwhitespace=false,         
    breaklines=true,                 
    captionpos=b,                    
    keepspaces=true,                 
    numbers=left,                    
    numbersep=5pt,                  
    showspaces=false,                
    showstringspaces=false,
    showtabs=false,                  
    tabsize=2
}
 
\lstset{style=mystyle}

\begin{document}
   \section{Fumadores - Lukas Häring}
   		\subsection{Comentarios}
   		Encontramos en éste ejemplo (n+2) semáforos, vamos a comentar cada uno de ellos
   		\begin{enumerate}
   			\item En la línea 13, encontramos un semáforo para el estanquero, el cual tiene valor inicial 1, pues éste debe producir al principio, éste es llamado con un \textbf{sem\_signal} cuando un consumidor aleatorio es elegido por la  \textbf{función\_hebra\_estanquero} haciéndo que dicho consumidor empiece a consumir (Cambiando su signal a 1). Además el estanquero no producirá un producto hasta que que no sea pedido por un fumador tras haber consumido su producto en un tiempo aleatorio.
   			\item En la línea 14, encontramos "n" semáforos, pero en éste ejemplo están asignados 3 semáforos, éstos se encargan de controlar a los fumadores y comenzarán estando desactivados (Valor 0) pues no hay estanquero que les haya dado un producto, es decir estarán esperando hasta que el estanquero les haga un sem\_signal aleatoriamente y así ellos poder realizar la función \textbf{fumar()} y finalmente volver a llamar al estanquero para que vuelva a producir.
   			\item La línea 15 contiene un semáforo que actúa de \textit{mutex} (Valor inicial en 1), éste es usado para poder realizar llamadas al sistema (Como por ejemplo escribir en la consola) y que dos comentarios no se interpongan, pues así conseguimos una homogeneidad en los comentarios de cada una de las funciones, no se caracteriza por nada en concreto.
   		\end{enumerate}
   		\subsection{Código}
   		\begin{lstlisting}[language=C++]
#include <iostream>
#include <vector>
#include <cassert>
#include <thread>
#include <random> // dispositivos, generadores y distribuciones aleatorias
#include <chrono> // duraciones (duration), unidades de tiempo
#include "Semaphore.h"

using namespace std ;
using namespace SEM ;

const int CONSUMIDORES = 3;
Semaphore estanquero(1);
vector<Semaphore> consumidores;
Semaphore mutex_fuma(1);
//**********************************************************************
// plantilla de funcion para generar un entero aleatorio uniformemente
// distribuido entre dos valores enteros, ambos incluidos
// (ambos tienen que ser dos constantes, conocidas en tiempo de compilacion)
//----------------------------------------------------------------------

template< int min, int max > int aleatorio(){
	static default_random_engine generador( (random_device())() );
	static uniform_int_distribution<int> distribucion_uniforme( min, max ) ;
	return distribucion_uniforme( generador );
}

//----------------------------------------------------------------------
// funcion que ejecuta la hebra del estanquero
void funcion_hebra_estanquero(){
	while(true){
		estanquero.sem_wait();

		// Mutex nos permite escribir el cout
		int num_fumador = aleatorio<0,CONSUMIDORES-1>();
		mutex_fuma.sem_wait();
		cout << endl << "Produce para el fumador: " << num_fumador << endl;
		mutex_fuma.sem_signal();
		consumidores[num_fumador].sem_signal();
	}
}

//-------------------------------------------------------------------------
// Funcion que simula la accion de fumar, como un retardo aleatoria de la hebra
void fumar(int num_fumador){

	// calcular milisegundos aleatorios de duracion de la accion de fumar)
	chrono::milliseconds duracion_fumar( aleatorio<20,200>() );

	// informa de que comienza a fumar
	mutex_fuma.sem_wait();
	cout << "Fumador " << num_fumador << " empieza a fumar (" << duracion_fumar.count() << " milisegundos)" << endl;
	mutex_fuma.sem_signal();

	// espera bloqueada un tiempo igual a ''duracion_fumar' milisegundos
	this_thread::sleep_for(duracion_fumar);

	// informa de que ha terminado de fumar
	mutex_fuma.sem_wait();
	cout << "Fumador " << num_fumador << "  : termina de fumar, comienza espera de ingrediente." << endl;
	mutex_fuma.sem_signal();

}

//----------------------------------------------------------------------
// funcion que ejecuta la hebra del fumador
void  funcion_hebra_fumador(int num_fumador){
	while(true){
		consumidores[num_fumador].sem_wait();
		mutex_fuma.sem_wait();
		cout << "Consume";
		mutex_fuma.sem_signal();
		fumar(num_fumador);
		estanquero.sem_signal();
	}
}

//----------------------------------------------------------------------

int main(){
	// declarar hebras y ponerlas en marcha
	thread hebra_productora(funcion_hebra_estanquero);
	vector<thread> hebra_consumidora;

	// Declaramos semaforos en "rojo" para cada consumidor
	// pues estos no pueden consumir si al principio no hay nada producido.
	for(int f = 0; f < CONSUMIDORES; f++){ consumidores.push_back(Semaphore(0)); }

	// Creamos una hebra para cada consumidor.
	for(int f = 0; f < CONSUMIDORES; f++){ hebra_consumidora.push_back(thread(funcion_hebra_fumador, f)); }

	// Anhadimos en la cola de ejecucion.
	for (auto& hebra : hebra_consumidora){
		hebra.join();
	}
	hebra_productora.join();

	return 0;
}

\end{lstlisting}

\end{document}