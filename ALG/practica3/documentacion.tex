\documentclass[12pt,a4paper]{report}
\usepackage[english]{babel}
\usepackage[utf8]{inputenc}
\usepackage{fancyhdr}
\usepackage{hyperref}
\usepackage{array}
\usepackage{amsfonts}
%math
\usepackage{amsmath}
\usepackage{mathtools}
\providecommand{\myceil}[1]{\left \lceil #1 \right \rceil }
\providecommand{\myfloor}[1]{\left \lfloor #1 \right \rfloor }

% graphics
\usepackage{graphicx}
\graphicspath{ {images/} }
% date
\usepackage{datetime}
\newdateformat{specialdate}{\THEYEAR-\twodigit{\THEMONTH}-\twodigit{\THEDAY}}
\date{\specialdate\today}
\usepackage{listings}
\usepackage{xcolor}
\lstset
{ %Formatting for code in appendix
	language=Matlab,
	basicstyle=\footnotesize,
	numbers=left,
	stepnumber=1,
	showstringspaces=false,
	tabsize=2,
	breaklines=true,
	breakatwhitespace=false,
}
\usepackage[miktex]{gnuplottex}  %% I am using miktex

% table of contents
\usepackage{tocloft}

%new commands
\usepackage{sectsty}
\newcommand\tab[1][1cm]{\hspace*{#1}}
\newcommand{\mychapter}[2]{
	\setcounter{chapter}{#1}
	\setcounter{section}{0}
	\chapter*{#2}
	\addcontentsline{toc}{chapter}{#2}
}
\chaptertitlefont{\LARGE}

%renew commands
\renewcommand{\cftchapleader}{\cftdotfill{\cftdotsep}}
\addto\captionsenglish{% Replace "english" with the language you use
	\renewcommand{\contentsname}%
	{Tabla de contenidos}%
}

\begin{document}
	\begin{titlepage}
		\centering
		\includegraphics[width=0.2\textwidth]{logo-ugr.png}\\*
		{\scshape\LARGE Universidad de Granada \par}
		{\large \date{\specialdate\today}\par}
		\vspace{1cm}
		{\LARGE\bfseries Divide y Vencerás\par}
		\vspace{1.5cm}
		{\scshape\large Algorítmica\par}
		\vspace{2cm}
		{\Large\itshape Lukas Häring García 2ºD\par}
	\end{titlepage}
	\tableofcontents
	
	\mychapter{0}{Suma hasta un número}
	Se trata de un ejercicio de obtener un sub-conjunto cuya suma de los elementos sumen el número que buscamos, la definición matemática es la siguiente.\\*\\*
	Sea $\textbf{S} = \{a_1, a_2, \dots a_n\}$ y $\textbf{M}$ como el sumando total.\\*
	Definimos el conjunto $\textbf{K}$ como aquél formado por todos los subconjuntos tal que sus elementos sumen $\textbf{M}$ 
	\[
		K = \left\{\{b_1,b_2,\dots, b_k\}\subseteq S \mid \sum_{i=1}^{k}b_i=M \right\}
	\]
	Ahora bien, pueden ocurrir dos casos:
	\begin{enumerate}
		\item No tenga solución el problema, es decir $K=\emptyset$.\\*
		Entonces devolvemos aquel subconjunto que supone el elemento minimal al problema.
		\[
			G = \left\{\exists G_1\forall G_2 \subseteq \mathcal{P}(S) \mid G_1\neq G_2 \wedge \left( \vert M - \sum_{x\in G_1}x \vert < \vert M-\sum_{y\in G_2}y \vert \right) \right\}
		\]
		\textit{$\mathcal{P}$ es denotado como el "conjunto potencia" de un conjunto}.
		\item Si tiene solución, es decir $\vert K \vert \ge 1$, devolvemos aquel subconjunto de $K$ que nuestro algoritmo encontró primero, puede o no ser con operadores mínimos.
	\end{enumerate}
	\newpage
	\section{Eficiencia}
	Para cada elemento en el conjunto de los números S, le asignamos "si" o "no" si queremos que se sume o no, esto equivale a multiplicar por 1 o 0 (respectivamente), denotemos $c_k$ para el elemento $a_k\in S$.
	\begin{center}
		
		\begin{tabular}{ | p{1cm} | p{1cm} | p{1cm} | p{4.5cm} | p{1cm} | p{1cm} | p{1cm} | }\hline
		\centering	$c_1$ &\centering $c_2$ &\centering $c_3$ &\centering  ... &\centering $c_{n-2}$ &\centering $c_{n-1}$ & \centering $c_n$\arraybackslash  \\\hline
		\end{tabular}
	\end{center}
	Definimos $f\colon S \rightarrow \mathbb{B}_2$, una asignación a cada elemento de $S$ un único elemento de $\mathbb{B}_2=\{0,1\}$, esta por tanto, es una aplicación inyectiva, cuyo número de aplicaciones es $\vert \mathbb{B}_2\vert^{\vert S\vert}=2^{\vert S\vert}$, por lo que el algoritmo usado será \textbf{exponencial} de la forma $O(2^n)$, dónde $n=\vert S \vert$.
	\mychapter{3}{Especificaciones}
	\begin{enumerate}
		\item Windows 10.0.14393
		\item Procesador Intel(R) Core(TM) i7-7800X CPU @ 3.50GHz, 3504 Mhz
		\item 6 procesadores principales.
		\item 12 procesadores lógicos.
		\item Memoria física instalada (RAM) 8,00 GB x 2
		\item Compilador MinGW.
	\end{enumerate}

\end{document}