\documentclass[12pt,a4paper]{report}
\usepackage[english]{babel}
\usepackage[utf8]{inputenc}
\usepackage{fancyhdr}
\usepackage{hyperref}

% graphics
\usepackage{graphicx}
\graphicspath{ {images/} }
% date
\usepackage{datetime}
\newdateformat{specialdate}{\THEYEAR-\twodigit{\THEMONTH}-\twodigit{\THEDAY}}
\date{\specialdate\today}
\usepackage{listings}
\usepackage{xcolor}
\lstset
{ %Formatting for code in appendix
	language=Matlab,
	basicstyle=\footnotesize,
	numbers=left,
	stepnumber=1,
	showstringspaces=false,
	tabsize=2,
	breaklines=true,
	breakatwhitespace=false,
}
\usepackage[miktex]{gnuplottex}  %% I am using miktex

% table of contents
\usepackage{tocloft}

%new commands
\newcommand\tab[1][1cm]{\hspace*{#1}}
\newcommand{\mychapter}[2]{
	\setcounter{chapter}{#1}
	\setcounter{section}{0}
	\chapter*{#2}
	\addcontentsline{toc}{chapter}{#2}
}

%renew commands
\renewcommand{\cftchapleader}{\cftdotfill{\cftdotsep}}
\addto\captionsenglish{% Replace "english" with the language you use
	\renewcommand{\contentsname}%
	{Tabla de contenidos}%
}

\begin{document}
	\begin{titlepage}
		\centering
		\includegraphics[width=0.2\textwidth]{logo-ugr.png}\\*
		{\scshape\LARGE Universidad de Granada \par}
		{\large \date{\specialdate\today}\par}
		\vspace{1cm}
		{\LARGE\bfseries Algoritmos de Ordenación con estructuras jerárquicas\par}
		\vspace{1.5cm}
		{\scshape\large Algorítmica\par}
		\vspace{2cm}
		{\Large\itshape Lukas Häring García 2ºD\par}
	\end{titlepage}
	\tableofcontents
	\mychapter{0}{Árbol binario de búsqueda}
	\section{Código}
	\begin{lstlisting}[language=c++]
using namespace std;
	
int main(int argc, char * argv[])
{
	int n = atoi(argv[1]);
	int* A = new int[n];
	ABB<int> ab_bus;
	srand(time(0));
	// Introducimos
	for (int i = 0; i < n; i++){
		ab_bus.Insertar(rand());
	}
	
	ABB<int>::nodo k;
	int m = 0;
	clock_t t_antes = clock();
	
	/*
		Generar codigo O(1) que el compilador no va a eliminar o
		desenrrollar.
	*/
	for (k = ab_bus.begin(); k != ab_bus.end(); ++k){
		 
		A[m++] = *k;
	}
	clock_t t_despues = clock();
	cout << n << "  " << ((double)(t_despues - t_antes)) / CLOCKS_PER_SEC << endl;
	
	return 0;
};
	\end{lstlisting}
	\newpage
	\section{Eficiencia}
	\subsection{Teórica}
		Para analizar el código, debemos suponer cuál es la forma de dicha estructura tras introducir datos, a simple vista, este parece tener $O(n)$, pero en realidad, hay que observar el operador incremento y el método end(Tras observar el código, vemos que es O(1)):
		\begin{enumerate}
			\item \textbf{El peor caso}. La altura del árbol es la misma que el número de elementos: Por tanto, este deberá bajar hasta la última posición, coste $O(n)$ y luego recorrerlo hacia atrás, por lo que también sería $O(n)$.
			\item \textbf{El mejor caso}. Esto ocurre cuando hay números en diferentes alturas, este tendrá que bajar como máximo a una altura ${log}_2(d)$ donde $d$ es la altura, pero al tener que recorrer todos los elementos, $O({log}_2(n) + n) \in O(n)$
		\end{enumerate}
		Concluimos que su eficiencia es $\Theta(n)$ pues ambos (peor y mejor caso) coinciden.
	\subsection{Empírica}
	Realizando el algoritmo de ordenación por inserción desde 0 dando pasos de 100 hasta 10000 elementos y realizando un ajuste con gnuplot, calculando las variables ocultas, obtenemos una eficiencias de:\\*
	$$f(n)=-1.51427\cdot 10^{-9}\cdot n + 2.73733\cdot 10^{-5}\in \Theta(n)$$
	\newpage
	\section{Gráficas}
	\subsection{Teórica}
	\begin{figure}[h!]
		\begin{gnuplot}[terminal=pdf,terminaloptions=color]
			plot [0:10] x title 'Orden Θ'
		\end{gnuplot}
	\end{figure}
	\subsection{Empírica}
	\begin{figure}[h!]
		\begin{gnuplot}[terminal=pdf,terminaloptions=color]
			f(x)=-1.51427e-009*x+2.73733e-005
			plot "./graph-data/tree_abb.dat" title 'Orden Empírico Θ', f(x) title 'adjustment'
		\end{gnuplot}
	\end{figure}

	\mychapter{1}{Árbol parcialmente ordenado}
	\section{Código}
	\begin{lstlisting}[language=c++]
int main(int argc, char * argv[])
{
	int n = atoi(argv[1]);
	int* A = new int[n];
	APO<int> apo_tree;
	srand(time(0));
	// Introducimos
	for (int i = 0; i < n; i++){
		apo_tree.insertar(rand());
	}
	
	int m = 0;
	clock_t t_antes = clock();
	while(!apo_tree.vacio()){
		apo_tree.borrar_minimo();
	}
	clock_t t_despues = clock();
	cout << n << "  " << ((double)(t_despues - t_antes)) / CLOCKS_PER_SEC << endl;
	
	return 0;
};
	\end{lstlisting}
	\newpage
	\section{Eficiencia}
	\subsection{Teórica}
	De la misma forma, el recorrer el arbol es $O(n)$, pero quién tiene importancia son los métodos insertar, que son más complejos.
	\subsection{Empírica}
	Realizando el algoritmo de ordenación por selección desde 0 dando pasos de 100 hasta 10000 elementos y realizando un ajuste con gnuplot, calculando las variables ocultas, obtenemos una eficiencias de:
	$$f(n)=1.5138\cdot 10^{-7}\cdot n + 0.000529994 \in \Theta(n)$$
	\newpage
	\section{Gráficas}
	\subsection{Teórica}
	\begin{figure}[h!]
		\begin{gnuplot}[terminal=pdf,terminaloptions=color]
			plot [0:10] x title 'Orden Θ'
		\end{gnuplot}
	\end{figure}
	\subsection{Empírica}
	\begin{figure}[h!]
		\begin{gnuplot}[terminal=pdf,terminaloptions=color]
			f(x)=1.5138e-007*x+0.000529994
			plot "./graph-data/tree_apo.dat" title 'Orden Empírico Θ', f(x) title 'adjustment'
		\end{gnuplot}
	\end{figure}

	\mychapter{3}{Especificaciones}
	\begin{enumerate}
		\item Windows 10.0.14393
		\item Procesador Intel(R) Core(TM) i7-7800X CPU @ 3.50GHz, 3504 Mhz
		\item 6 procesadores principales.
		\item 12 procesadores lógicos.
		\item Memoria física instalada (RAM) 8,00 GB x 2
		\item Compilador MinGW.
	\end{enumerate}

\end{document}