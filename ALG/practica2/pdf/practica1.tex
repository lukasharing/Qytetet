\documentclass[12pt,a4paper]{report}
\usepackage[english]{babel}
\usepackage[utf8]{inputenc}
\usepackage{fancyhdr}
\usepackage{hyperref}
%math
\usepackage{amsmath}
\usepackage{mathtools}
\providecommand{\myceil}[1]{\left \lceil #1 \right \rceil }
\providecommand{\myfloor}[1]{\left \lfloor #1 \right \rfloor }

% graphics
\usepackage{graphicx}
\graphicspath{ {images/} }
% date
\usepackage{datetime}
\newdateformat{specialdate}{\THEYEAR-\twodigit{\THEMONTH}-\twodigit{\THEDAY}}
\date{\specialdate\today}
\usepackage{listings}
\usepackage{xcolor}
\lstset
{ %Formatting for code in appendix
	language=Matlab,
	basicstyle=\footnotesize,
	numbers=left,
	stepnumber=1,
	showstringspaces=false,
	tabsize=2,
	breaklines=true,
	breakatwhitespace=false,
}
\usepackage[miktex]{gnuplottex}  %% I am using miktex

% table of contents
\usepackage{tocloft}

%new commands
\usepackage{sectsty}
\newcommand\tab[1][1cm]{\hspace*{#1}}
\newcommand{\mychapter}[2]{
	\setcounter{chapter}{#1}
	\setcounter{section}{0}
	\chapter*{#2}
	\addcontentsline{toc}{chapter}{#2}
}
\chaptertitlefont{\LARGE}

%renew commands
\renewcommand{\cftchapleader}{\cftdotfill{\cftdotsep}}
\addto\captionsenglish{% Replace "english" with the language you use
	\renewcommand{\contentsname}%
	{Tabla de contenidos}%
}

\begin{document}
	\begin{titlepage}
		\centering
		\includegraphics[width=0.2\textwidth]{logo-ugr.png}\\*
		{\scshape\LARGE Universidad de Granada \par}
		{\large \date{\specialdate\today}\par}
		\vspace{1cm}
		{\LARGE\bfseries Divide y Vencerás\par}
		\vspace{1.5cm}
		{\scshape\large Algorítmica\par}
		\vspace{2cm}
		{\Large\itshape Lukas Häring García 2ºD\par}
	\end{titlepage}
	\tableofcontents
	\mychapter{0}{Máximo y mínimo en un vector}
	\section{Código}
	\begin{lstlisting}[language=c++]
#include<utility>
#include<limits>
#include<cmath>
// Suponemos que n > 0
// PAR RESULTANTE: (MAX, MIN)
std::pair<int, int> Max_Min(int* v, int n){
	std::pair<int, int> result(*v, *v);
	// Si solo hay un elemento, entonces ese es el minimo y maximo
	if(n == 1){
		return result;
	}
	// Si hay dos elementos, entonces el min y el max son ambos.
	if(n == 2){
		if(*(v + 0) < *(v + 1)){
			result.first = *(v + 1);
		}else{
			result.second = *(v + 1);
		}
		return result;
	}
	// Calculamos las particiones (Segun sea par o impar).
	int bt = floor(n / 2.0), tp = ceil(n / 2.0);
	// Particion izquierda.
	std::pair<int, int> left = Max_Min(v, bt);
	// Particion derecha.
	std::pair<int, int> right = Max_Min(v + bt, tp);
	
	// max{max0, max1}, min{min0, min1}
	result.first = std::max(left.first, right.first);
	result.second = std::min(left.second, right.second);
	// Devolvemos el resultado.
	return result;
}
	\end{lstlisting}
	\newpage
	\section{Eficiencia (n = Tamaño de la muestra)}
	\[   
		T(n) = 
		\begin{cases}
			1 & n = 1, 2\\
			T\left(\myfloor{\dfrac{n}{2}}\right) + T\left(\myceil{\dfrac{n}{2}}\right) + 1 & n \ge 3 \\
		\end{cases}
	\]
	Realizando la siguiente sustitución, $n=2^k$.\\*
	$$T(2^k)=T(\myfloor{2^{k - 1}}) + T(\myceil{2^{k - 1}}) + 1=2\cdot T(2^{k-1}) + 1$$
	\begin{enumerate}
		\item La parte \textbf{homogénea} es: $T(2^k)-2\cdot T(2^{k-1})$.
		\item La parte \textbf{no homogénea} es: $1=1\cdot 1^k$.
	\end{enumerate}
	Por lo que resolviendo las ecuaciones, obtenemos que:
	\begin{enumerate}
		\item De la parte \textbf{homogénea} es: $k-2=0\Rightarrow k=2$.
		\item De la parte \textbf{no homogénea}: $k=1$.
	\end{enumerate}
	Obtenemos la ecuación no recurrente.\\*
	$$T(2^k)=2^k\cdot c_1+1^k\cdot c_2=n\cdot c_1+c_2=T(n)\Rightarrow O(T(n))=O(n\cdot c_1 + c_2)\in O(n)$$
	\section{Gráfica}
	\begin{figure}[h!]
		\begin{gnuplot}[terminal=pdf,terminaloptions=color]
			f(x)=3.31988e-008*x-8.77114e-005
			plot "./datos/maximominimo.dat" title 'Orden Empírico', f(x) title 'adjustment'
		\end{gnuplot}
	\end{figure}
	\mychapter{1}{Mínimo y máximo en una matriz}
	\section{Código}
	\begin{lstlisting}[language=c++]
#include<utility>
#include<limits>
#include<cmath>
// PAR RESULTANTE: (MAX, MIN)
std::pair<int,int> Max_Min(int** v,int ii,int jj,int n){
	std::pair<int, int> result;
	if(n == 2){ // Matriz 2 x 2
		int max = std::numeric_limits<int>::min();
		int min = std::numeric_limits<int>::max();
		for(int j = jj; j < jj + n; ++j){
			for(int i = ii; i < ii + n; ++i){
				min = std::min(v[j][i], min);
				max = std::max(v[j][i], max);
			}
		}
		result.first = max;
		result.second = min;
		return result;
	}
	// Calculamos la la siguiente particion.
	int k = n % 2, tp = ceil(n / 2.0);
	int rr = tp - k;
	std::pair<int,int> tl = Max_Min(v, ii     , jj     , tp);
	std::pair<int,int> tr = Max_Min(v, ii + rr, jj     , tp);
	std::pair<int,int> bl = Max_Min(v, ii     , jj + rr, tp);
	std::pair<int,int> br = Max_Min(v, ii + rr, jj + rr, tp);
	//max,min{(max0, min0),(max1, min1),(max2, min2),(max3, min3)}
	result.first = std::max(std::max(tl.first, tr.first),
	std::max(bl.first, br.first));
	result.second = std::min(std::min(tl.second, tr.second),
	std::min(bl.second, br.second));
	// Devolvemos el resultado.
	return result;
}
		\end{lstlisting}
		\newpage
		\section{Eficiencia ($n^2$ = Tamaño de la matriz)}
		\[   
			T(n^2) = 
			\begin{cases}
			1 & n = 2\\
			4\cdot T\left(\dfrac{n^2}{4}\right) + 1 & n \ge 3 \\
			\end{cases}
		\]
		Realizando la siguiente sustitución, $n^2=2^k$.\\*
		$$T(2^k)=4\cdot T(2^{k-2}) + 1=0\cdot T(2^{k-1}) + 4\cdot T(2^{k-2}) + 1$$
		\begin{enumerate}
			\item La parte \textbf{homogénea} es: $T(2^k) - 4\cdot T(2^{k-2})$.
			\item La parte \textbf{no homogénea} es: $1=1\cdot 1^k$.
		\end{enumerate}
		Por lo que resolviendo las ecuaciones, obtenemos que:
		\begin{enumerate}
			\item De la parte \textbf{homogénea} es: $k^2-4=0\Rightarrow k=\sqrt{4}=2$.
			\item De la parte \textbf{no homogénea}: $k=1$.
		\end{enumerate}
		Obtenemos la ecuación no recurrente (Conociendo que $2^k = n^2$).\\*
		$$T(2^k)=2^k\cdot c_1+1^k\cdot c_2=T(n^2)\Rightarrow O(T(n^2))=O(n^2\cdot c_1 + c_2)\in O(n^2)$$
		\section{Gráfica}
		\begin{figure}[h!]
			\begin{gnuplot}[terminal=pdf,terminaloptions=color]
				f(x)=3.61796e-008*x*x+-8.95664e-005*x+0.122892
				plot "./datos/maximominimomatriz.dat" title 'Orden Empírico', f(x) title 'adjustment'
			\end{gnuplot}
		\end{figure}
	
		\mychapter{2}{Zapatos con sus pies}
		\section{Código}
		\begin{lstlisting}[language=c++]
#include<iostream>
#include<utility>
#include <ctime>

// Realiza el swap de dos valores en un vector.
void swap(int* v, int i, int j){
	int h = *(v + i);
	*(v + i) = *(v + j);
	*(v + j) = h;
}

// Desplaza los elementos desde la posicion i de un vector v 
// e introduce en la posicion i, el valor k
void shift(int* v, int i, int n, int k){
	for(int j = n - 1; j > i; --j){
		*(v + j) = *(v + j - 1);
	}
	*(v + i) = k;
}
\end{lstlisting}
\newpage
\begin{lstlisting}[language=c++]
void Zapatos_Pies(int* shoes, int* toes, int n){
	if(n <= 1){
		return;
	}
	// Pivote (i) para los pies del ultimo zapato
	int i = 0;
	// Buscamos el pivote en para los pies.
	for(i = 0; *(toes + i) != *(shoes + n - 1); ++i);
	// Hacemos swap del pie pivote con el ultimo
	swap(toes, i, n - 1);

	int m, w;
	/* MOVEMOS LOS PIES */
	m = -1;
	w = 0;
	while(w < n){
		// Comparamos el elemento con el pivote de los pies
		// Si este es menor, intercambiamos el muro con el actual.
		if(*(toes + w) < *(shoes + n - 1)){
			++m;
			swap(toes, m, w);
		}
		++w;
	}
	// Movemos los elementos a la derecha.
	shift(toes, ++m, n, *(shoes + n - 1));

	/* MOVEMOS LOS ZAPATOS */
	m = -1;
	w = 0;
	while(w < n){
		// Comparamos el elemento con el pivote de los pies
		// Si este es menor, intercambiamos el muro con el actual.
		if(*(shoes + w) < *(toes + n - 1)){
			++m;
			swap(shoes, m, w);
		}
		++w;
	}
	// Movemos los elementos a la derecha.
	shift(shoes, ++m, n, *(toes + n - 1));
	
	// Ahora los nuevos pivotes son m0, m1
	// Repetimos el algoritmo
	Zapatos_Pies(shoes, toes, m);
	Zapatos_Pies(shoes + m, toes, n - m);
}
		\end{lstlisting}
		\newpage
		\section{Eficiencia}
		La eficiencia de los métodos \textit{swap} y \textit{shift} son $O(1)$ y $O(n)$, respectivamente.
		Realizamos un quicksort para cada conjunto, para los pies y para los zapatos, como referencia el pivote del conjunto opuesto.
		Definimos como $n$ el total de pies/niños y $k$ cantidad de elementos mayores al pivote.
		\[   
		T(n) = 
		\begin{cases}
		1 & n \le 1\\
		n + T(n-k-1) + T(k) & n \ge 2 \wedge 0 \le k \le n \\
		\end{cases}
		\]
		Según la regla de simetría. \[k=0,n-1\Rightarrow T(n-0-1)+T(0)=T(n-(n-1)-1)+T(n-1)=T(0)+T(n-1)\]
		Aplicando lo dicho, 
		\[
			T(n)=n+T(0)+T(n-1)=n+1+T(n-1)
		\]
		\[
			T(n)-T(n-1)=0\Rightarrow x-1=0, 1^n(n+1)
		\]
		El polinomio resultante y por la regla de simetría:
		\[
			T(n)=1^n(n^2c_1+nc_2+c_3)\Rightarrow \Theta(T(n))=\Theta(n^2c_1+nc_2+c_3)\in \Theta(n^2)
		\]
		\section{Gráfica}
		\begin{figure}[h!]
			\begin{gnuplot}[terminal=pdf,terminaloptions=color]
				f(x)=5.26918e-009*x*x-3.93483e-008*x+8.30631e-005
				plot "./datos/zapatospies.dat" title 'Orden Empírico', f(x) title 'adjustment'
			\end{gnuplot}
		\end{figure}
	
	\mychapter{3}{Moda de un conjunto de enteros}
	\section{Código}
	\begin{lstlisting}[language=c++]
std::pair<int, int> Mayor_Frecuencia(int* set, int n){
	int* ht0 = new int[n];
	int* ht1 = new int[n];
	int p = set[0]; // PIVOTE (Elemento inicial al vector).
	int h0 = 0, h1 = 0, m = 1;
	// Si el tamanio es <= 1
	if(n <= 1){
		return std::pair<int, int>(m, p);
	}
	// Metemos segun la homogeneidad de los elementos.
	for(int i = 1; i < n; ++i){
		int df = p - set[i];
		if(df > 0){
			ht0[h0] = set[i];
			++h0;
		}else if(df < 0){
			ht1[h1] = set[i];
			++h1;
		}else{
			++m;
		}
	}
	std::pair<int, int> v0 = Mayor_Frecuencia(ht0, h0);
	std::pair<int, int> v1 = Mayor_Frecuencia(ht1, h1);
	delete[] ht0;
	delete[] ht1;
	// Buscamos en los dos conjuntos heterogeneos (Aquel mayor).
	if(m > v0.first && m > v1.first){
		return std::pair<int, int>(n, p);
	}else if(v0.first > m && v0.first > v1.first){
		return v0;
	}else{
		return v1;
	}
}
	\end{lstlisting}
	\newpage
	\section{Eficiencia}
	Donde $n$ es el tamaño de la muestra, $k$ las repeticiones al pivote y $d$ los elementos mayores al pivote.
	\[   
	T(n) = 
	\begin{cases}
	1 & n \le 1\\
	T(n-k-d) + T(d) + n & n \ge 2 \wedge 1 \le d < k \le n \\
	\end{cases}
	\]
	Diferenciamos dos casos
	\begin{enumerate}
		\item Cuando todos los elementos solo se repiten una vez y no se divide.
		\[
			k=1, d=0\Rightarrow T(n)=T(n-1)+T(0)+n=T(n-1)+1+n
		\]
		Es parecida al ejercicio anterior, por lo que será $O(n^2)$.
		\item $k=n$ Cuando no hay elementos mayores y las repeticiones como n.
		\[
		k=n, d=0\Rightarrow T(n)=T(n-n-0)+T(0)+n=2T(0)+n=n\in \Omega(n)
		\]
		
	\end{enumerate}
	
	\section{Gráfica}
	\begin{figure}[h!]
		\begin{gnuplot}[terminal=pdf,terminaloptions=color]
			f(x)=-4.57096e-014*x*x+3.33947e-008*x+0.000106985+0.0001
			g(x)=3.20187e-008*x+0.000113866-0.0001
			plot "./datos/mayorfreq.dat" title 'Orden Empírico Θ', f(x) title 'Upper bound', g(x) title 'Lower bound'
		\end{gnuplot}
	\end{figure}

	\mychapter{3}{Especificaciones}
	\begin{enumerate}
		\item Windows 10.0.14393
		\item Procesador Intel(R) Core(TM) i7-7800X CPU @ 3.50GHz, 3504 Mhz
		\item 6 procesadores principales.
		\item 12 procesadores lógicos.
		\item Memoria física instalada (RAM) 8,00 GB x 2
		\item Compilador MinGW.
	\end{enumerate}

\end{document}