\documentclass[12pt,a4paper]{report}
\usepackage[english]{babel}
\usepackage[utf8]{inputenc}
\usepackage{fancyhdr}
\usepackage{hyperref}
%math
\usepackage{amsmath}
\usepackage{mathtools}
\providecommand{\myceil}[1]{\left \lceil #1 \right \rceil }
\providecommand{\myfloor}[1]{\left \lfloor #1 \right \rfloor }

% graphics
\usepackage{graphicx}
\graphicspath{ {images/} }
% date
\usepackage{datetime}
\newdateformat{specialdate}{\THEYEAR-\twodigit{\THEMONTH}-\twodigit{\THEDAY}}
\date{\specialdate\today}
\usepackage{listings}
\usepackage{xcolor}
\lstset
{ %Formatting for code in appendix
	language=Matlab,
	basicstyle=\footnotesize,
	numbers=left,
	stepnumber=1,
	showstringspaces=false,
	tabsize=2,
	breaklines=true,
	breakatwhitespace=false,
}
\usepackage[miktex]{gnuplottex}  %% I am using miktex

% table of contents
\usepackage{tocloft}

%new commands
\usepackage{sectsty}
\newcommand\tab[1][1cm]{\hspace*{#1}}
\newcommand{\mychapter}[2]{
	\setcounter{chapter}{#1}
	\setcounter{section}{0}
	\chapter*{#2}
	\addcontentsline{toc}{chapter}{#2}
}
\chaptertitlefont{\LARGE}

%renew commands
\renewcommand{\cftchapleader}{\cftdotfill{\cftdotsep}}
\addto\captionsenglish{% Replace "english" with the language you use
	\renewcommand{\contentsname}%
	{Tabla de contenidos}%
}

\begin{document}
	\begin{titlepage}
		\centering
		\includegraphics[width=0.2\textwidth]{logo-ugr.png}\\*
		{\scshape\LARGE Universidad de Granada \par}
		{\large \date{\specialdate\today}\par}
		\vspace{1cm}
		{\LARGE\bfseries Divide y Vencerás\par}
		\vspace{1.5cm}
		{\scshape\large Algorítmica\par}
		\vspace{2cm}
		{\Large\itshape Lukas Häring García 2ºD\par}
	\end{titlepage}
	\tableofcontents
	\mychapter{0}{Ejercicio 1}
	Sea $a[1..n]$, $n\ge1$, un vector de enteros diferentes y ordenados crecientemente, tal que algunos de los
	valores pueden ser negativos. Diseñar un algoritmo que devuelva un índice natural k, $1\le k\le n$,
	talque $a[k]=k$, siempre que tal índice exista. El coste del algoritmo debeser $O(\log{n})$ en el caso peor.
	Se pide diseñar el algoritmo dando todo lujo de detalles. Justificar que funciona correctamente y el
	coste. Repetir el problema pero para un vector de n enteros posiblemente repetidos y ordenado
	decrecientemente.
	
	\begin{lstlisting}[language=c++]
int abs(int k){ return (k < 0 ? -k : k); }
int sign(int k){ return(k == 0 ? 0 : (k / abs(k))); }
int k_posicion_k(int* v, int a, int b){
	if(b < a) return -1;
	
	int m = (a + b) / 2, l = v[m];
	// Miramos en uno de los dos lados.
	switch(sign(m - l)){
		case 1:  m = k_posicion_k(v, m + 1, b); break;
		case -1: m = k_posicion_k(v, a, m - 1); break;
	}
	return m;
}
	\end{lstlisting}
	\newpage
	\mychapter{1}{Ejercicio 2}
	Dados n enteros cualesquiera $a_1,a_2,...,a_n$, necesitamos encontrar el valor de la expresión:
	\[
		\max_{0\le i\le j<n}{\sum_{k=i}^{j}a_k}
	\]
	que calcula el máximo de las sumas parciales de elementos consecutivos.
	Deseamos implementar un algoritmo Divide y Vencerás de complejidad $n\log n$ que resuelva el
	problema. ¿Existe algún otro algoritmo que lo resuelva en menor tiempo?

	\newpage
	\mychapter{2}{Ejercicio 4}
	Se dispone de un conjunto de $n$ tornillos de diferente tamaño y sus correspondientes $n$ tuercas, de
	forma que cada tornillo encaja perfectamente con una y sólo una tuerca. Dado un tornillo y una
	tuerca, uno es capaz de determinar si el tornillo es menor que la tuerca, mayor, o encaja
	exactamente. Sin embargo, no hay forma de comparar dos tornillos o dos tuercas entre ellos para
	decidir un orden. Se desea ordenar los dos conjuntos de forma que los elementos que ocupan la
	misma posición en los dos conjuntos emparejen entre sí.
	
\begin{lstlisting}[language=c++]
// Realiza el swap de dos valores en un vector.
void s(int* v,int i,int j){int h=*(v+i);*(v+i)=*(v+j);*(v+j)=h;}
void shift(int* v, int i, int n, int k){
	for(int j=n-1;j>i;--j){*(v+j)=*(v+j-1);}*(v+i)=k;}
void Tornillos_Tuercas(int* tornillos, int* tuercas, int n){
	if(n <= 1){ return; }
	int i = 0;
	for(i = 0; *(tuercas + i) != *(tornillos + n - 1); ++i);
	s(tuercas, i, n - 1);//SWAP
	int m = -1, w = 0;
	while(w < n){ /* MOVEMOS LAS TUERCAS */
		if(*(tuercas + w) < *(tornillos + n - 1)){
			++m;
			s(tuercas, m, w);//SWAP
		}++w;
	}
	shift(tuercas, ++m, n, *(tornillos + n - 1));
	m = -1; w = 0;
	while(w < n){ /* MOVEMOS LOS TORNILLOS */
		if(*(tornillos + w) < *(tuercas + n - 1)){
			s(tornillos, m++, w);//SWAP
		}++w;
	}
	shift(tornillos, ++m, n, *(tuercas + n - 1));
	Tornillos_Tuercas(tornillos, tuercas, m);
	Tornillos_Tuercas(tornillos + m, tuercas, n - m);
}
\end{lstlisting}
\newpage
\mychapter{5}{Ejercicio 5}
Dado un vector de números enteros positivos y negativos, encontrar el subvector de casillas
contiguas que contenga la secuencia en orden creciente no monótono de longitud máxima. Por
ejemplo, dado el vector \{2, -1, 3, 3, 8, 7, 6, -2, 5\}, la secuencia creciente no monótona de longitud
máxima es $\{-1, 3, 3, 8\}$. En caso de que haya dos secuencias distintas con el mismo tamaño,
bastaría con obtener una de ellas

\begin{lstlisting}[language=c++]
int* mayor_serie(int* v, int a, int b){
	int diff = b - a + 1;
	int* res = NULL;
	switch(diff){
		case 1: res = new int[1]{ v[a] }; break;
		case 2:
			if(v[a] >= v[b]){ res = new int[2]{ v[a], v[b] }; }
		break;
		default:
			int m = (a + b) / 2;
			int* s1 = mayor_serie(v, a, m - 1);
			int* s2 = mayor_serie(v, m, b);
			if(s1 != NULL && s2 != NULL){
				if(s1[m - 1 - a] <= s2[0]){
					res = new int[diff];
					for(int i = 0; i < (m - 1 - a); ++i){
						res[i] = s1[i];
					}
					for(int i = 0; i < (b - m); ++i){
						res[m + i] = s2[i];
					}
				}
				delete[] s1, s2;
			}else if(s1 != NULL){ delete[] s2; res = s1; 
			}else(s2 != NULL){ delete[] s1; res = s2; }
		break;
	}
	return res;
}
\end{lstlisting}
	\mychapter{3}{Especificaciones}
	\begin{enumerate}
		\item Windows 10.0.14393
		\item Procesador Intel(R) Core(TM) i7-7800X CPU @ 3.50GHz, 3504 Mhz
		\item 6 procesadores principales.
		\item 12 procesadores lógicos.
		\item Memoria física instalada (RAM) 8,00 GB x 2
		\item Compilador MinGW.
	\end{enumerate}
\end{document}