\documentclass[12pt,a4paper]{report}
\usepackage[english]{babel}
\usepackage[utf8]{inputenc}
\usepackage{fancyhdr}
\usepackage{hyperref}
%math
\usepackage{amsmath}
\usepackage{mathtools}
\providecommand{\myceil}[1]{\left \lceil #1 \right \rceil }
\providecommand{\myfloor}[1]{\left \lfloor #1 \right \rfloor }

% graphics
\usepackage{graphicx}
\graphicspath{ {images/} }
% date
\usepackage{datetime}
\newdateformat{specialdate}{\THEYEAR-\twodigit{\THEMONTH}-\twodigit{\THEDAY}}
\date{\specialdate\today}
\usepackage{listings}
\usepackage{xcolor}
\lstset
{ %Formatting for code in appendix
	language=Matlab,
	basicstyle=\footnotesize,
	numbers=left,
	stepnumber=1,
	showstringspaces=false,
	tabsize=2,
	breaklines=true,
	breakatwhitespace=false,
}
\usepackage[miktex]{gnuplottex}  %% I am using miktex

% table of contents
\usepackage{tocloft}

%new commands
\usepackage{sectsty}
\newcommand\tab[1][1cm]{\hspace*{#1}}
\newcommand{\mychapter}[2]{
	\setcounter{chapter}{#1}
	\setcounter{section}{0}
	\chapter*{#2}
	\addcontentsline{toc}{chapter}{#2}
}
\chaptertitlefont{\LARGE}

%renew commands
\renewcommand{\cftchapleader}{\cftdotfill{\cftdotsep}}
\addto\captionsenglish{% Replace "english" with the language you use
	\renewcommand{\contentsname}%
	{Tabla de contenidos}%
}

\begin{document}
	\begin{titlepage}
		\centering
		\includegraphics[width=0.2\textwidth]{logo-ugr.png}\\*
		{\scshape\LARGE Universidad de Granada \par}
		{\large \date{\specialdate\today}\par}
		\vspace{1cm}
		{\LARGE\bfseries Divide y Vencerás\par}
		\vspace{1.5cm}
		{\scshape\large Algorítmica\par}
		\vspace{2cm}
		{\Large\itshape Lukas Häring García 2ºD\par}
	\end{titlepage}
	\tableofcontents
	\mychapter{0}{Ejercicio 1}
	Deseamos almacenar en una cinta magnética de longitud L un conjunto de $n$ programas ${P_1, P_2, \dots , P_n}$. Sabemos que cada $P_i$ necesita un espacio $a_i$ de la cinta y que $(\sum a_i) > L$, $1\le i \le n$. Construir un
	algoritmo que seleccione aquel subconjunto de programas que hace que el número de programas almacenado en la cinta sea máximo.
	\begin{enumerate}
		\item \textbf{Conjunto de Candidatos}.\newline
		Todos los programas.
		\item \textbf{Función de Selección}.\newline
		Seleccionamos aquel proceso que ocupe menos.
		\item \textbf{Función Factible}.\newline
		Si el espacio anteriormente ocupado $ + a_k \le L$. 
		\item \textbf{Objetivo}.\newline
		Maximizar el número de procesos que entran en la cinta.
		\item \textbf{Solución}.\newline
		Se devuelve un vector $V=\{i_1, \cdots, i_k\}$ con todos los índices (sin importar el orden) de cada uno de los procesos.
	\end{enumerate}
	En el peor de los caso, cuando el vector de procesos no está ordenado por tamaño, la eficiencia tiende a un orden $O(n^2)$ tras tener que comprar cada elemento con el anterior.\newline
	\textbf{Formulación matemática}.
	\[
		\max\left\{\sum^k_{j=1}P_{i_k}\right\}\le L
	\]
	\newpage
	\mychapter{1}{Ejercicio 3}
	Se han de procesar $n$ tareas con instantes de terminación $t_i$ y tiempo de proceso $p_i$. Se dispone de
	un procesador y las tareas no son interrumpibles. Una solución factible planifica todas las tareas de
	modo que terminan en, o antes de, su instante asignado. Diseñar un algoritmo voraz para encontrar
	la solución. 
	\begin{enumerate}
		\item \textbf{Conjunto de Candidatos}.\newline
		Todas las tareas.
		\item \textbf{Función de Selección}.\newline
		La tarea que tenga menor tiempo de ejecución y sea factible.
		\item \textbf{Función Factible}.\newline
		Comprueba que si en el instante actual $a$ y el tiempo que tarda en ejecutarse el proceso $p_i$ es menor al tiempo de terminación $a + p_i < t_i$.
		\item \textbf{Objetivo}.\newline
		Ejecutar el máximo de tareas antes del instante asignado.
		\item \textbf{Solución}.\newline
		Un vector $V=\{i_1, \cdots, i_k\}$ que contiene el índice de la tarea a asignar (Cola).
	\end{enumerate}
	La eficiencia del algoritmo es $O(n^2)$ ya que debe buscar todo aquel tiempo de proceso mínimo y que el instante actual sea inferior al de terminación más el de ejecución.\newline
	\textbf{Formulación matemática}.
	
	
	\newpage
	\mychapter{1}{Ejercicio 4}
	Dado un conjunto de $n$ cintas cuyo contenido está ordenado y con ni registros cada una, se han de
	mezclar por pares hasta lograr una única cinta ordenada. La secuencia en que se haga la mezcla
	determina la eficiencia del proceso. Diseñar un algoritmo voraz que minimice el número de movimientos.
	\begin{enumerate}
		\item \textbf{Conjunto de Candidatos}.\newline
		Todas las cintas.
		\item \textbf{Función Factible}.\newline
		No tiene.
		\item \textbf{Función de Selección}.\newline
		Se selecciona aquella cinta no utilizada anteriormente y menor a todas las demás que tampoco han sido utilizadas. 
		\item \textbf{Objetivo}.\newline
		Minimizar el número de movimientos entre las cintas.
		\item \textbf{Solución}.\newline
		Una cola ordenada según que elemento se coge y el siguiente es aquel que se vierte.
	\end{enumerate}
	Como hay insertar todo aquel elemento maximal a los insertados anteriormente, su eficiencia es $O(n^2)$.

	\newpage
	\mychapter{2}{Ejercicio 5}
	Un automovilista ha de ir con su vehículo desde la población A hasta la población B siguiendo una
	ruta prefijada (por ejemplo la carretera N-340). Con el depósito completamente lleno puede hacer
	$X$ km. El conductor conoce de antemano en qué lugares de la carretera existen gasolineras (la
	distancia entre dos gasolineras consecutivas es inferior a $X$ km.). Diseñar un algoritmo que permita
	que el automovilista vaya de A a B repostando el mínimo número de veces posible (se supone que
	parte de A con el depósito lleno y que el coche no se averiará, ni tendrá un accidente, ni será
	abducido, etc. durante el trayecto).Indicar el coste del algoritmo
	\begin{enumerate}
		\item \textbf{Candidatos}.\newline
		Todas aquellas gasolineras que están en el camino del automovilista.
		\item \textbf{Función de Selección}.\newline
		Seleccionamos aquella gasolinera que está más lejos de las demás alcanzables por el coche y en la dirección correcta.
		\item \textbf{Función Factible}.\newline
		Verifica si la gasolinera es alcanzable con la gasolina restante.
		\item \textbf{Objetivo}.\newline
		Pasar por el mínimo número de gasolineras.
		\item \textbf{Solución}.\newline
		Un vector de gasolineras por donde el conductor debe pasar.
	\end{enumerate}
	Si existen $n$ gasolineras y cada una está a la distancia $(X - 1)$km de la anterior, por lo que debemos pasar por todas las ellas, es decir, $O(n)$.
\newpage
\mychapter{5}{Ejercicio 7}
Dado un grafo $G=<N,A>$ siendo $N$ el conjunto de vértices y $A$ el conjunto de aristas. Se pide
colorear el grafo de forma que si dos vértices están unidos por una arista los colores asignados a
estos vértices tienen que ser distintos. El objetivo en este problema es usar el menor numero de
colores para colorear todos los vértices del grafo. 
	\begin{enumerate}
		\item \textbf{Conjunto de Candidatos}.\newline
		Todos los vértices del grafo.
		\item \textbf{Función de Selección}.\newline
		Se seleccionan un vértice no coloreado, cualquiera.
		\item \textbf{Función Factible}.\newline
		Verifica si el color asignado no coincide con los que tienen sus vecinos.
		\item \textbf{Objetivo}.\newline
		Usar el menor número de colores.
		\item \textbf{Solución}.\newline
		Un vector de colores para cada vértice, tal que el vértice $v_i$ tiene un color $c_i$.
	\end{enumerate}
	Cuando el grafo (de $n$ vértices) tiene el mismo número de vértices que de vecinos (esto ocurre cuando es un grafo completo $K_n$), se debe comprobar el color con el de los vecinos. La eficiencia tiende a $O(n^2)$.
	\mychapter{3}{Especificaciones}
	\begin{enumerate}
		\item Windows 10.0.14393
		\item Procesador Intel(R) Core(TM) i7-7800X CPU @ 3.50GHz, 3504 Mhz
		\item 6 procesadores principales.
		\item 12 procesadores lógicos.
		\item Memoria física instalada (RAM) 8,00 GB x 2
		\item Compilador MinGW.
	\end{enumerate}
\end{document}