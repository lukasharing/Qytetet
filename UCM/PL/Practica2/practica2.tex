\documentclass{article}

%packages
\usepackage{graphicx}
\usepackage{tikz}
\usepackage[utf8]{inputenc}
\usepackage{amssymb}
\usepackage{listings}
\begin{document}

\usetikzlibrary{automata,arrows, positioning}
\renewcommand{\contentsname}{Tabla de contenidos}

\begin{titlepage}
	\begin{center}
		\includegraphics{escudo.jpg}
	\end{center}
	\centering
	{\scshape\LARGE Complutense de Madrid \par}
	\vspace{1cm}
	{\scshape\Large Práctica Procesadores de Lenguajes.\par}
	\vspace{1.5cm}
	{\huge\bfseries Segunda Fase. \par}
	\vspace{2cm}
	{\Large\itshape Kyle Tan \& Lukas Häring\par}
	{\large Grupo 9\par}
	\vfill
	\vfill

% Bottom of the page
	{\large \today\par}
\end{titlepage}

\tableofcontents
\newpage
\section{Diagrama sintáctico}

\tikzstyle{b} = [rectangle, draw, text width=5em, text centered, minimum height=3em]
\tikzstyle{more} = [rectangle, draw, text width=8em, text centered, minimum height=3em, dashed]
\tikzstyle{ini} = [circle, draw, fill=black]
\tikzstyle{l} = [draw, -latex']

\begin{tikzpicture}[>=stealth',shorten >=0.5pt,auto,node distance=5 cm, scale = 0.85, transform shape]

\node[ini]   (initial) at(0.5, 0.0) {};
\node[state] (LIT) at(7.0, 0.0) {LIT};
\node[state] (+) at(7.0, -1.5) {+};
\node[state] (-) at(7.0, -3.0) {-};
\node[state] (*) at(7.0, -4.5) {*};
\node[state] (/) at(7.0, -6.0) {/};
\node[state] (not) at(4.0, -9.0) {not};
\node[state] (mns) at(4.0, -10.5) {-};
\node[more] (rel) at(7.0, -7.5) {Operadores Relacionales};
\node [b]    (exp1) at(3.5, -4.5) {exp};
\node [b]    (exp2) at(10.5, -4.5) {exp};
\node [b]    (exp3) at(9.5, -9.75) {exp};
\node[ini]    (final) at(13.5, 0.0) {};
\node[state] (POP) at(3.5, -12.0) {(};
\node[state] (PCL) at(10.5, -12.0) {)};
\node [b]    (exp4) at(7.0, -12.0) {exp};


\draw[black, dashed] (0.0, 0.5) rectangle (14,-12.5);
\node[align=center,text width=0.5cm] at (0.5, 0.3) {exp};

% Initial
\path [l] (initial)  -- (LIT);
\path [l] (initial)  -- (1.0, 0.0) -- (1.0, -4.5) -- (exp1);
\path [l] (initial)  -- (1.0, 0.0) -- (1.0, -9.75) -- (2.5, -9.75) -- (2.5, -9.0) -- (not);
\path [l] (initial)  -- (1.0, 0.0) -- (1.0, -9.75) -- (2.5, -9.75) -- (2.5, -10.5) -- (mns);
\path [l] (initial)  -- (1.0, 0.0) -- (1.0, -12.0) -- (POP);

\path [l] (exp1)  -- (5.0, -4.5) -- (5.0, -1.5) -- (+);
\path [l] (exp1)  -- (5.0, -4.5) -- (5.0, -3.0) -- (-);
\path [l] (exp1)  -- (5.0, -4.5) -- (5.0, -4.5) -- (*);
\path [l] (exp1)  -- (5.0, -4.5) -- (5.0, -6.0) -- (/);
\path [l] (exp1)  -- (5.0, -4.5) -- (5.0, -7.5) -- (rel);
\path [l] (+)  -- (9.0, -1.5) -- (9.0, -4.5) -- (exp2);
\path [l] (-)  -- (9.0, -3.0) -- (9.0, -4.5) -- (exp2);
\path [l] (*)  -- (9.0, -4.5) -- (9.0, -4.5) -- (exp2);
\path [l] (/)  -- (9.0, -6.0) -- (9.0, -4.5) -- (exp2);
\path [l] (rel) -- (9.0, -7.5) -- (9.0, -4.5) -- (exp2);
\path [l] (not)  -- (5.5, -9.0) -- (5.5, -9.75) -- (exp3);
\path [l] (mns)  -- (5.5, -10.5) -- (5.5, -9.75) -- (exp3);
\path [l] (POP) -- (exp4);
\path [l] (exp4) -- (PCL);

% Final 
\path [l] (LIT)  -- (final);
\path [l] (exp2) -- (12.5, -4.5) -- (12.5, 0.0) -- (final);
\path [l] (exp3) -- (12.5, -9.75) -- (12.5, 0.0) -- (final);
\path [l] (PCL) -- (12.5, -12.0) -- (12.5, 0.0) -- (final);

\end{tikzpicture}
\newline
\begin{tikzpicture}[>=stealth',shorten >=0.5pt,auto,node distance=5 cm, scale = 0.85, transform shape]

\node[ini]   (initial) at(0.5, -0.75) {};
\node [state]    (num) at(5.5, 0.0) {num};
\node [state]    (bool) at(5.5, -1.5) {bool};
\node [state]    (PYC) at(6.5, -3.0) {;};
\node [state]    (id) at(10.5, -0.75) {ID};
\node[ini]    (final) at(13.5, -0.75) {};


\draw[black, dashed] (0.0, 0.5) rectangle (14,-3.5);
\node[align=center,text width=2.2cm] at (1.2, 0.0) {bloque inicialización};

\path [l] (num)  -- (8.5, 0.0) -- (8.5, -0.75) -- (id);
\path [l] (bool)  -- (8.5, -1.5) -- (8.5, -0.75) -- (id);

\path [l] (initial)  -- (2.5, -0.75) -- (2.5, 0.0) -- (num);
\path [l] (initial)  -- (2.5, -0.75) -- (2.5, -1.5) -- (bool);


\path [l] (12.0, -0.75) -- (12.0, -3.00) -- (PYC);
\path [l] (PYC) -- (1.5, -3.00) -- (1.5, -0.75);



\path [l] (id)  -- (final);

\end{tikzpicture}
\newline
\begin{tikzpicture}[>=stealth',shorten >=0.5pt,auto,node distance=5 cm, scale = 0.85, transform shape]

\node[ini]   (initial) at(0.5, 0.0) {};
\node [state]    (num) at(3.5, 0.0) {ID};
\node [state]    (EQ) at(7.0, 0.0) {=};
\node [state]    (PYC) at(7.0, -1.5) {;};
\node [state]    (id) at(10.5, 0.0) {exp};
\node[ini]    (final) at(13.5, 0.0) {};


\draw[black, dashed] (0.0, 0.75) rectangle (14,-2.0);
\node[align=center,text width=2.75cm] at (1.5, 0.52) {bloque asignación};

\path [l] (num) -- (EQ);
\path [l] (EQ) -- (id);

\path [l] (initial) -- (num);

\path [l] (12.0, 0.0) -- (12.0, -1.50) -- (PYC);
\path [l] (PYC) -- (1.5, -1.50) -- (1.5, 0.0);

\path [l] (id)  -- (final);

\end{tikzpicture}
\newline
\begin{tikzpicture}[>=stealth',shorten >=0.5pt,auto,node distance=5 cm, scale = 0.85, transform shape]

\node[ini]   (initial) at(0.5, 0.0) {};
\node [b]    (num) at(3.5, 0.0) {bloque inicialización};
\node [state]    (EQ) at(7.0, 0.0) {\&\&};
\node [b]    (id) at(10.5, 0.0) {bloque asignación};
\node[ini]    (final) at(13.5, 0.0) {};


\draw[black, dashed] (0.0, 0.75) rectangle (14,-0.75);
\node[align=center,text width=1cm] at (0.75, 0.52) {Programa};

\path [l] (num) -- (EQ);
\path [l] (EQ) -- (id);

\path [l] (initial) -- (num);

\path [l] (id)  -- (final);

\end{tikzpicture}
\newpage

\section{Gramática Incontextual con prioridades}

$
\begin{array}{lcl}
	E_0 & \rightarrow & E_0\quad O_0\quad E_1 \\
	E_0 & \rightarrow & E_1 \\
	O_0 & \rightarrow & + \\
	O_0 & \rightarrow & -\quad (binario) \\
	E_1 & \rightarrow & E_2\quad O_{11}\quad E_1 \\
	E_1 & \rightarrow & E_2\quad O_{12}\quad E_2 \\
	E_1 & \rightarrow & E_2 \\
	O_{11} & \rightarrow & and \\
	O_{12} & \rightarrow & or \\
	E_2 & \rightarrow & E_3\quad O_2\quad E_3 \\
	E_2 & \rightarrow & E_3 \\
	O_2 & \rightarrow & Operadores\quad Relacionales \\
	E_3 & \rightarrow & E_3\quad O_3\quad E_4 \\
	E_3 & \rightarrow & E_4 \\
	O_3 & \rightarrow & * \\
	O_3 & \rightarrow & / \\
	E_4 & \rightarrow & -\quad O_{41}\quad E_4 \\
	E_4 & \rightarrow & -\quad O_{42}\quad E_5 \\
	E_4 & \rightarrow & (\quad E_0\quad ) \\
	E_4 & \rightarrow & Literal \\
	O_{41} & \rightarrow & -\quad (unario) \\
	O_{42} & \rightarrow & not \\
\end{array}
$

\newpage

\section{Gramática LL(1)}

$
\begin{array}{lcl}
E_0 & \rightarrow & E_1\quad E_0'  \\
E_0' & \rightarrow & \epsilon \\
E_0' & \rightarrow & O_o\quad E1\quad E_0' \\
O_0 & \rightarrow & + \\
O_0 & \rightarrow & -\quad (binario) \\
E_1 & \rightarrow & E_2\quad E_1' \\
E_1' & \rightarrow & \epsilon \\
E_1' & \rightarrow & O_{11}\quad E_{1}\\
E_1' & \rightarrow & O_{12}\quad E_{2}\\
O_{11} & \rightarrow & and \\
O_{12} & \rightarrow & or \\
E_2 & \rightarrow & E_3\quad E_2'\\
E_2' & \rightarrow & \epsilon \\
E_2' & \rightarrow & O_2\quad E_3 \\
O_2 & \rightarrow & Operadores\quad relacionales \\
E_3 & \rightarrow & E_4\quad E_3' \\
E_3' & \rightarrow & O_3\quad E_4\quad E_3'\\
E_3' & \rightarrow & \epsilon \\
O_3 & \rightarrow & * \\
O_3 & \rightarrow & / \\
E_4 & \rightarrow & O_{41}\quad E_4 \\
E_4 & \rightarrow & O_{42}\quad E_5 \\
O_{41} & \rightarrow & - \\
O_{42} & \rightarrow & not \\
E_4 & \rightarrow & E_5 \\
E_5 & \rightarrow & (\quad E_5 \quad) \\
E_5 & \rightarrow & Literal \\
\end{array}
$

\newpage 

\section{Primeros y siguientes}

Podemos obtener los "primeros y siguientes" utilizando la herramienta \textbf{Proletool}. Para ello hemos escrito el siguiente código.

\lstset{frame=tb,
	aboveskip=3mm,
	belowskip=3mm,
	showstringspaces=false,
	columns=flexible,
	basicstyle={\small\ttfamily},
	numbers=none,
	breaklines=true,
	breakatwhitespace=true,
	tabsize=3
}

\begin{lstlisting}
grammar recursiva_der
{
	nonterminal E0, E0p;
	nonterminal E1, E1p;
	nonterminal E2, E2p;
	nonterminal E3, E3p;
	nonterminal E4;
	nonterminal E5;
	nonterminal O0, O11, O12, O2, O3, O41, O42; 
	
	terminal literal;
	terminal operadores_relacionadas; 
	terminal and, or, not;
	
	E0  := E1 E0p; 
	E0p  := O0 E1 E0p |  ; 
	O0 := '+' | '-'; 
	
	E1 := E2 E1p; 	
	E1p := O11 E1 | O12 E2 |  ; 
	O11 := and; 
	O12 := or;
	
	E2 := E3 E2p; 
	E2p := O2 E3 |  ; 
	O2 := operadores_relacionadas; 
	
	E3 := E4 E3p; 
	E3p := O3 E4 E3p |  ; 
	O3 := '*' | '/'; 
	
	E4 := O41 E4 | O42 E5 | E5; 
	O41 := '-'; 	
	O42 := not; 
	
	E5 := '(' E0 ')' | literal; 
} 
\end{lstlisting}

\newpage
\subsection{Primeros}
\begin{tabular}{|l|l|}
	\hline
	\multicolumn{1}{|c|}{No terminal} & \multicolumn{1}{c|}{Iniciales} \\ \hline
O0 &	+ -\\ \hline
E2p &	operadores\_relacionadas\\ \hline
O2 &	operadores\_relacionadas\\ \hline
E3p &	* /\\ \hline
E0p &	+ -\\ \hline
O3 &	* /\\ \hline
E1p &	or and\\ \hline
E0 &	not ( - literal\\ \hline
E1 &	not ( - literal\\ \hline
O42 &	not\\ \hline
E2 &	not ( - literal\\ \hline
O41 &	-\\ \hline
O11 &	and\\ \hline
E3 &	not ( - literal\\ \hline
E4 &	not ( - literal\\ \hline
E5 &	( literal\\ \hline
O12 &	or\\ \hline
\end{tabular}

\subsection{Siguientes}

\begin{tabular}{|l|l|}
	\hline
	\multicolumn{1}{|c|}{No terminal} & \multicolumn{1}{c|}{Seguidores} \\ \hline
	O0 &	not ( - literal \\ \hline
	E2p &	or \$ and ) + - \\ \hline
	O2 &	not ( - literal \\ \hline
	E3p &	operadores\_relacionadas or \$ and ) + - \\ \hline
	E0p &	\$ ) \\ \hline
	O3 &	not ( - literal \\ \hline
	E1p &	\$ ) + - \\ \hline
	E0 &	\$ ) \\ \hline
	E1 &	\$ ) + - \\ \hline
	O42 &	( literal \\ \hline
	E2 &	or \$ and ) + - \\ \hline
	O41 &	not ( - literal \\ \hline
	O11 &	not ( - literal \\ \hline
	E3 &	operadores\_relacionadas or \$ and ) + - \\ \hline
	E4 &	operadores\_relacionadas or \$ and ) * + - / \\ \hline
	E5 &	operadores\_relacionadas or \$ and ) * + - / \\ \hline
	O12 &	not ( - literal \\ \hline
\end{tabular}

\end{document}
