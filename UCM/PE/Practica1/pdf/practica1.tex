\documentclass{article}

%packages
\usepackage{graphicx}
\usepackage{tikz}
\usepackage[utf8]{inputenc}
\usepackage{amssymb}

\begin{document}

\usetikzlibrary{automata,arrows, positioning}
\renewcommand{\contentsname}{Tabla de contenidos}

\begin{titlepage}
	\begin{center}
		\includegraphics{escudo.jpg}
	\end{center}
	\centering
	{\scshape\LARGE Complutense de Madrid \par}
	\vspace{1cm}
	{\scshape\Large Práctica Programación Evolutiva.\par}
	\vspace{1.5cm}
	{\huge\bfseries Primera práctica. \par}
	\vspace{2cm}
	{\Large\itshape Raúl Torrijos \& Lukas Häring\par}
	%{\large Grupo 9\par}
	\vfill
	\vfill

% Bottom of the page
	{\large \today\par}
\end{titlepage}

\tableofcontents


\section{Introducción}
En esta práctica vamos a realizar un analizador léxico sencillo, un lenguaje inventado, donde solo existen dos tipos, \textbf{num} y \textbf{bool}, por un lado, los números pueden ser enteros con signo o reales (tanto decimales como exponenciales) (estos números pueden estár precedidos de ilimitado número de ceros). Las operaciones básicas: multiplicación, división, adición y substracció; Operaciones de comparación, como son menor, mayor, menor igual y mayor igual. Por otro lado los booleanos, pueden ser o \textbf{true} o \textbf{false}, sus operadores característicos son el \textit{and}, \textit{or} y el \textit{not}. Siendo todo lo anteriormente dicho, palabras reservadas. Los operadores aceptan paréntesis para alterar las precedencias.

\newpage

\section{Definición}
\subsection{Clases Auxiliares}
\begin{center}
    \begin{tabular}{ | p{2cm} | p{5cm} | p{4cm} |}
    \hline
    Auxiliares & Definición Natural & Expresión Regular \\ \hline
    Letra & Carácter del alfabeto mayúsculas o minúsculas  & [a-z,A-Z]\\ \hline
    Dígito & Número entero de 0 a 9 & [0-9] \\ \hline
    Natural & Número entero positivo & Dígito* \\ \hline
    \end{tabular}
\end{center}
\subsection{Clases Léxicas}
\begin{center}
    \begin{tabular}{ | p{2cm} | p{5cm} | p{4cm} |}
    \hline
     Clases Léxicas & Definición Natural & Expresión Regular \\ \hline
     ID & Identificador & Letra(Letra$\mid$Dígito$\mid$\_)$*$ \\ \hline
     INT & Número entero & (+$\mid$-)?Natural \\ \hline
     DEC & Número decimal &  INT.Natural \\ \hline
     EXE & Número exponencial  & (INT$\mid$DEC)(e$\mid$E)?INT  \\ \hline
     PLS & Signo más & + \\ \hline
     MNS & Signo menos & - \\ \hline
     MUL & Signo multiplicación & * \\ \hline
     DIV & Signo división & / \\ \hline
     PAO & Paréntesis abierto & ( \\ \hline
      PCE & Paréntesis cerrado & ) \\ \hline
      GT & Signo mayor que & $>$\\ \hline
      GE & Signo mayor igual que & $>=$ \\ \hline
      LS & Signo menor que & $<$\\ \hline
      LG & Signo menor igual que & $<=$ \\ \hline
      EQ & Signo de asignación & = \\ \hline
      BE & Signo de comparación & == \\ \hline
      NE & Signo de desigualdad & !=  \\ \hline
			SEP & Separador de definición e inicialización & \&\& \\ \hline
			PYC & Separador de variables y asignación de variables & $;$ \\ \hline
    \end{tabular}
\end{center}
\subsection{Clases Ignorables}
\begin{center}
    \begin{tabular}{ | p{2cm} | p{5cm} | p{4cm} |}
    \hline
    Clases Ignorables & Definición Natural & Expresión Regular \\ \hline
    VAC & Carácteres ignorables & EOL$\mid \emptyset$ \\ \hline
    \end{tabular}
\end{center}

\newpage

\section{Diagrama de transiciones}

\begin{tikzpicture}[>=stealth',shorten >=0.5pt,auto,node distance=5 cm, scale = 0.85, transform shape]


\node[initial,state]    (A) at (0,0) {INICIO};
\node[state]  (B) at(1.5, 5.0) {AND};
\node[state,accepting]  (B1) at(4.0, 5.5) {SEP};
\node[state,accepting]  (C) at(-1.5, 5.0) {PAO};
\node[state,accepting]  (D) at(-3.0, 4.5) {DIV};
\node[state,accepting]  (M) at(-4.0, 3.5) {MUL};
\node[state,accepting]  (E) at(6.0, 0.0) {ID};
\node[state,accepting]  (F) at(0.0, 5.5)      {PCE};
\node[state,accepting] (H1) at(3.0, 4.5) {GT};
\node[state,accepting] (GE) at(5.0, 4.5) {GE};
\node[state,accepting] (H2) at(4.0, 3.5) {LT};
\node[state,accepting] (LG) at(6.0, 3.5) {LE};
\node[state,accepting] (EQ) at(5.0, 2.5) {EQ};
\node[state,accepting] (BE) at(7.0, 2.5) {BE};
\node[state] (N) at(5.5, 1.5) {N};
\node[state,accepting] (NE) at(7.5, 1.5) {NE};
\node[state,accepting] (PLS) at(5.0, -2.5) {PLS};
\node[state,accepting] (EN) at(3.5, -4.0) {INT};
\node[state,accepting] (MNS) at(1.5, -5.0) {MNS};
\node[state] (P) at(6.5, -5.0) {PNT};
\node[state] (EX) at(3.5, -7.0) {EXP};
\node[state,accepting] (DC) at(6.5, -7.0) {DEC};
\node[state] (SG) at(0.5, -7.0) {SGN};
\node[state,accepting] (XE) at(-2.5, -7.0) {EXE};
\node[state,accepting] (PY) at(0.0, -5.0) {PYC};
\node[state,accepting] (EOF) at(-1.5, -5.0) {EOF};

\path[->]
      (A) edge [left]  node [align=center]  {\&} (B)
      (B) edge [above]  node [align=center]  {\&} (B1)
			(A) edge [left]   node [align=center]  {$($} (C)
      (A) edge [left]  node [align=center]  {$/$} (D)
			(A) edge [left]  node [align=center]  {$* $} (M)
			(A) edge [above]  node [align=center]  {[a-z,A-Z]} (E)
			(E) edge [loop below]  node [align=center]  {([a-z,A-Z,0-9]$\mid$\_)} (E)
			(A) edge [above left]  node [align=center]  {$)$} (F)
			(A) edge [above left]  node [align=center]  {$>$} (H1)
			(A) edge [above left]  node [align=center]  {$<$} (H2)
			(H1) edge [above]  node [align=center]  {$=$} (GE)
			(H2) edge [above]  node [align=center]  {$=$} (LG)
			(A) edge [above]  node [align=center]  {$=$} (EQ)
			(EQ) edge [above]  node [align=center]  {$=$} (BE)
			(A) edge [above]  node [align=center]  {$!$} (N)
			(N) edge [above]  node [align=center]  {$=$} (NE)
			(A) edge [right]  node [align=center]  {$+$} (PLS)
			(PLS) edge [bend left]  node [align=center]  {[0-9]} (EN)
			(A) edge [right]  node [align=center]  {$-$} (MNS)
			(MNS) edge [bend right]  node [align=center]  {[0-9]} (EN)
			(EN) edge [loop above]  node [align=center]  {[0-9]} (EN)
			(A) edge [right]  node [align=center]  {[0-9]} (EN)
			(EN) edge [right]  node [align=center]  {e$\mid$E} (EX)
			(P) edge [left]  node [align=center]  {[0-9]} (DC)
			(DC) edge [loop right]  node [align=center]  {[0-9]} (DC)
			(DC) edge [above]  node [align=center]  {e$\mid$E} (EX)
			(EX) edge [above]  node [align=center]  {+$\mid$-} (SG)
			(SG) edge [above]  node [align=center]  {[0-9]} (XE)
			(XE) edge [loop above]  node [align=center]  {[0-9]} (XE)
			(EN) edge [above]  node [align=center]  {.} (P)
      (A) edge [left]  node [align=center]  {;} (PY)
			(A) edge [left]  node [align=center]  {EOF} (EOF)
			(EX) edge [bend left]  node [align=center]  {[0-9]} (XE)
			(A) edge [out=200,in=238,looseness=30]  node [align=center]  {$\emptyset \mid$ EOL} (A);
\end{tikzpicture}
\begin{center}
    \begin{tabular}{ | p{5cm} | p{5cm} |}
    \hline
    Identificador & Valor devuelto \\ \hline
		\textbf{num} & Devuelve numero \\ \hline
		\textbf{bool} & Devuelve booleano \\ \hline
		\textbf{and} & Devuelve operador binario and \\ \hline
		\textbf{or} & Devuelve operador binario or \\ \hline
		\textbf{not} & Devuelve operador unario not\\ \hline
		\textbf{true} & Devuelve verdadero \\ \hline
		\textbf{false} & Devuelve falso \\ \hline

    \end{tabular}
\end{center}
\end{document}
