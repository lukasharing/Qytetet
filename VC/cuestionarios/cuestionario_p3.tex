\documentclass[12pt,a4paper]{article}
\usepackage[english]{babel}
\usepackage[utf8]{inputenc}
\usepackage{fancyhdr}
\usepackage{hyperref}


% Math
\usepackage{bm}
\usepackage{amsmath} 

\begin{document}
	\begin{titlepage}
		\centering
		{\scshape\LARGE Universidad de Granada \par}
		{\large \date{\specialdate\today}\par}
		\vspace{1cm}
		{\LARGE\bfseries Cuestionario 3\par}
		\vspace{1.5cm}
		{\scshape\large Visión por computador\par}
		\vspace{2cm}
		{\Large\itshape Lukas Häring García\par}
	\end{titlepage}
	
	
	\newpage
	
	1. ¿Cuál es la transformación más fuerte de la geometría de una escena que puede introducir se al tomar una foto de ella? Dar algún ejemplo.
	\newpage
	2. Por qué es necesario usar el plano proyectivo para estudiar las transformaciones en las imágenes de fotos de escenas? Dar algún ejemplo.
	\newpage
	3. Sabemos que en el plano proyectivo un punto no existe en el sentido del plano afín,sino que se define por una clase de equivalencia de vectores definida por ${k(x,y,1),k\ne 0}$. Razone usando las coordenadas proyectivas de los puntos afines de una recta que pase por el $(0,0)$ del plano afín y verifique que los punto de la recta del infinito del plano proyectivo son necesariamente vectores del tipo $(*,*,0)$ con *=cualquier número.
	\newpage
	4. ¿Qué propiedades de la geometría de un plano quedan invariantes cuando se toma una foto de él? Justificar la respuesta.
	\newpage
	5. En  coordenadas  homogéneas  los  puntos y  rectas  del  plano  se representan  por  vectores  de  tres  coordenadas(notados  x  y  l respectivamente), de manera que si una recta contiene a un punto se verifica la ecuación $x^Tl=0$, es decir 
	$\begin{pmatrix}
		x1 & x2 & x3\\
	\end{pmatrix}
	\begin{pmatrix}
		x1 \\ x2 \\ x3\\
	\end{pmatrix} = 0$. 
	Considere una homografía H que transforma vectores de puntos, $x'=Hx$. Dado que una homografía transforma vectores de tres coordenadas también existen homografías G para transformar vectores de rectas $l'=Gl$.Suponga una recta l y un punto x que verifican $x^Tl=0$ en el plano proyectivo y suponga  que  conoce  una  homografía  H  que transforma vectores  de puntos. En estas condiciones ¿cuál es la homografía G que transforma los vectores de las rectas? Deducirla matemáticamente.
	\newpage
	6. ¿Cuál es el mínimo número de escalares necesarios para fijar 
	una homografía general? ¿Y si la homografía es afín?Justificar la respuesta
	\newpage
	7. Defina una homografía entre planos proyectivos que haga que el punto (3,0,2) del plano proyectivo-1 se transforme en un punto de la recta del infinito del plano proyectivo-2? Justificar la respuesta
	\newpage
	8. Una homografía general $H, det(H)\ne0$ admite una descomposición única en movimiento elementales de la siguiente forma $H=H_SH_AH_P$ donde $H_S$ representa la homografía de una similaridad (escala, giro y traslación), $H_A$ la homografía de un movimiento afín puro y  $H_P$ una transformación proyectiva pura.
	
	$$H = H_S \cdot H_A \cdot H_P$$
	$$
	\begin{pmatrix}
		a & b & c\\
		d & e & f\\
		g & h & i\\
	\end{pmatrix}
	= 
	\begin{pmatrix}
		s\cos(\theta) & -s\sin(\theta) & t_x\\
		s\sin(\theta) & s\cos(\theta) & t_y\\
		0 & 0 & 1\\
	\end{pmatrix}
	\cdot
	\begin{pmatrix}
		a & c & 0 \\
		0 & b & 0 \\
		0 & 0 & 1\\
	\end{pmatrix}
	 \cdot
	 \begin{pmatrix}
		 1 & 0 & 0 \\
		 0 & 1 & 0 \\
		 v_1 & v_2 & v\\
	 \end{pmatrix}
	$$
	$$
	\begin{pmatrix}
	a & b & c\\
	d & e & f\\
	g & h & i\\
	\end{pmatrix}
	= 
	\begin{pmatrix}
	as\cos(\theta) & cs\cos(\theta)-bs\sin(\theta) & t_x\\
	as\sin(\theta) & cs\sin(\theta)+bs\cos(\theta) & t_y\\
	0 & 0 & 1\\
	\end{pmatrix}
	\cdot
	\begin{pmatrix}
	1 & 0 & 0 \\
	0 & 1 & 0 \\
	v_1 & v_2 & v\\
	\end{pmatrix}
	$$
	$$
	\begin{pmatrix}
	a & b & c\\
	d & e & f\\
	g & h & i\\
	\end{pmatrix}
	= 
	\begin{pmatrix}
	as\cos(\theta) + t_xv_1 & cs\cos(\theta)-bs\sin(\theta)+t_xv_2 & t_xv\\
	as\sin(\theta) + t_yv_1 & cs\sin(\theta)+bs\cos(\theta)+t_yv_2 & t_yv\\
	v_1 & v_2 & v\\
	\end{pmatrix}
	$$
	\newline
	Vemos que la última fila ya está resuelta y tenemos que $v_1=g$, $v_2=h$ y $v=i$.
	Además, con $v=i$ podemos también resolver la última columna, por lo que $t_x=\frac{c}{i}$ y $t_y=\frac{f}{i}$.
	\newline
	Finalmente resolvemos para $\theta$ y $s$:
		$$ a = a\cdot s\cdot \cos(\theta)+t_xv_1\Rightarrow \dfrac{a-t_xv_1}{a}=\dfrac{ai-cg}{ai}=s\cdot cos(\theta)  $$
		$$ d = a\cdot s\cdot \sin(\theta)+t_yv_1\Rightarrow \dfrac{b-t_yv_1}{a}=\dfrac{di-fg}{ai}=s\cdot sin(\theta)  $$
	Dividimos el segundo con el primero y obtenemos:
		$$ \frac{s\cdot sin(\theta)}{s\cdot cos(\theta)} = tan(\theta) = \dfrac{\dfrac{di-fg}{ia}}{\dfrac{ai-cg}{ia}} = \dfrac{id-fg}{ia-cg}\Rightarrow\theta=\arctan\left(\dfrac{id-fg}{ia-cg}\right) $$
	Resolvemos para $scos(\theta)$, conociendo la siguiente igualdad $cos(arctan(x))=\dfrac{1}{\sqrt{x^2+1}}$
	\newline
		$$ s\cdot cos\left(\arctan\left(\dfrac{id-fg}{ia-cg}\right)\right)=\dfrac{s}{\sqrt{\left(\dfrac{id-fg}{ia-cg}\right)^2+1}}$$
		Finalmente,
		$$ s = \left(\dfrac{ai-cg}{ai}\right) \sqrt{\left(\dfrac{id-fg}{ia-cg}\right)^2+1} $$
	\newline
	
	Vamos ahora aplicar las ecuaciones obtenidas para resolver el siguiente ejemplo.
	
	$H=\begin{pmatrix}
		1.707 & 0.586 & 1.0\\
		2.707 & 8.242 & 2.0\\
		1.0 & 2.0 & 1.0\\
	\end{pmatrix}$
	
	$$
	\left\{
	\begin{array}{ll}
		v_1=g = 1.0\\
		v_2=h = 2.0\\
		v=i = 1.0\\
		t_x=\dfrac{c}{i}=\dfrac{1.0}{1.0}=1.0\\
		t_y=\dfrac{f}{i}=\dfrac{2.0}{1.0}=2.0\\
		\theta = \arctan\left(\dfrac{id-fg}{ia-cg}\right)=\arctan\left(\dfrac{2.707\cdot 1.0 - 2.0\cdot 1.0}{1.0\cdot 1.707-1.0\cdot 1.0}\right) = \arctan(1)=\dfrac{\pi}{4} \\
		
		s = \left(\dfrac{ai-cg}{ai}\right) \sqrt{\left(\dfrac{ib-fg}{ia-cg}\right)^2+1} = \left(1 - \dfrac{1.0\cdot 1.0}{1.707\cdot 1.0}\right)\sqrt{1^2+1}=\dfrac{707}{1707}\cdot\sqrt{2}
	\end{array}
	\right.
	$$
	\newline
	\textbf{Nota}: Si implementáramos estas equivalencias en un computador, tendremos pequeños errores de precisión.
	
	$
	\begin{pmatrix}
	1.707 & 0.586 & 1.0\\
	2.707 & 8.242 & 2.0\\
	1.0 & 2.0 & 1.0\\
	\end{pmatrix}
	= H_S\cdot H_A\cdot H_P
	$\newline
	Calculamos el seno y el coseno con los datos anteriores,\newline
	$s\cdot\sin(\theta)=\dfrac{707}{1707}\cdot\sqrt{2}\cdot \sin\left(\dfrac{\pi}{4}\right)=\dfrac{707}{1707}\cdot\sqrt{2}\cdot\dfrac{1}{\sqrt{2}}=\dfrac{707}{1707} = s\cdot\cos(\theta)$\newline
	$
	H_S=
	\begin{pmatrix}
	\dfrac{707}{1707} & -\dfrac{707}{1707} & 1.0\\
	\dfrac{707}{1707} & \dfrac{707}{1707} & 2.0\\
	0 & 0 & 1\\
	\end{pmatrix}
	$\newline
	$
	H_A=
	\begin{pmatrix}
	1.707 & 1.0 & 0 \\
	0 & 0.586 & 0 \\
	0 & 0 & 1.0\\
	\end{pmatrix}
	$\newline
	$
	H_P=
	\begin{pmatrix}
	1 & 0 & 0 \\
	0 & 1 & 0 \\
	1.0 & 2.0 & 1.0\\
	\end{pmatrix}
	$
	
	\newpage
	9. ¿Cuáles son las propiedades necesarias y suficientes para que una matriz defina un movimiento geométrico no degenerado entre planos? Justificar la respuesta
	\newpage
	
	10. ¿Qué información de la imagen usa el detector de Harris para seleccionar puntos? ¿El detector de Harris detecta patrones geométricos o fotométricos? Justificar la contestación.
	\newpage
	
	11. ¿Sería adecuado usar como descriptor de un punto Harris los valores de los píxeles de su región de soporte? Identifique ventajas, inconvenientes y mecanismos de superación de estos últimos.
	
	\newpage
	
	12. Describa un par de criterios que sirvan para seleccionar parejas de puntos en correspondencias ("matching") a partir de descriptores de regiones extraídos de dos imágenes. ¿Por qué no es posible garantizar que todas las parejas son correctas?
	
	\newpage
	13. Cual es el objetivo principal del uso de la técnica RANSAC en el cálculo de una homografía. Justificar la respuesta
	
	\newpage
	14. Si tengo 4 imágenes de una escena de manera que se solapan la 1-2, 2-3 y 3-4. ¿Cuál es el número mínimo de parejas de puntos en correspondencias necesarios para montar un mosaico? Justificar la respuesta
	
	\newpage
	15. ¿En la confección de un mosaico con proyección rectangular es esperable que aparezcan deformaciones geométricas de la escena real?¿Cuáles y por qué? ¿Bajo qué condiciones esas deformaciones podrían no estar presentes? Justificar la respuesta.
\end{document}